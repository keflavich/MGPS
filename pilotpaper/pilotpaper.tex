\documentclass[twocolumn]{aastex62}
%Following line instructs TeXShop to use latex + ghostscript:
%!TEX TS-program = latex
%\documentclass[12pt]{article}
%\documentclass{/Users/adam/papers/latexfiles/emulateapj}
%\documentclass{/Users/adam/papers/latexfiles/emulateapj}
%\usepackage{/Users/adam/papers/latexfiles/emulateapj5}
%\usepackage{/Users/adam/papers/latexfiles/deluxetable}

%\special{! /pdfmark 
%            [/View [/XYZ null null 1]  % unspecified x and y offset, 100% zoom
%             /Page 1
%             /PageModehttps://www.sharelatex.com/4759396695vhbmwsvkbydp /UseThumbs % /UseNone /UserOutlines /UseThumbs /FullScreen
%            /DOCVIEW pdfmark 
%            }

\usepackage[utf8]{inputenc}
\usepackage{natbib}  % Requires natbib.sty, available from http://ads.harvard.edu/pubs/bibtex/astronat/
\usepackage{rotating}
\usepackage{savesym}
\savesymbol{singlespace}
\savesymbol{doublespace}
\usepackage{wrapfig}
\usepackage{setspace}
\usepackage{xspace}
\usepackage{color}
\usepackage{multicol}
\usepackage{mdframed}
%\citestyle{aa}  % (Author YYYY) references instead of (Author, YYYY)
%\bibliographystyle{/Users/adam/papers/latexfiles/apj_w_etal}
%\bibliographystyle{/Users/adam/papers/latexfiles/apj_w_etal_3auth}
\newcommand{\red}[1]{\textcolor{red}{#1}}
\newcommand{\todo}[1]{\textcolor{red}{#1}}
\usepackage{enumitem}
\setlist{nolistsep}
\setlist{noitemsep}
%\setlist{nosep}
%\setlist{nopartopsep}
%\setlist{noparbottomsep}
\usepackage[compact]{titlesec}
\titlespacing{\section}{0pt}{8pt}{0pt}


%%% this achieves the holy grail of 1 in margins all around!!! %%%
	%%%%%%%%%%%%%%%%%%%%%%%%%%%%%%
%	\oddsidemargin  0.0in
%	\evensidemargin 0.0in
%	\textwidth      7in
%	\headheight     0.0in
%	\topmargin      1.00in
%	\textheight=9in
	%%%%%%%%%%%%%%%%%%%%%%%%%%%%%%


% some macros that will probably be useful...
\newcommand{\paa}{Pa\ensuremath{\alpha}}
\newcommand{\brg}{Br\ensuremath{\gamma}}
\newcommand{\msun}{\ensuremath{M_{\odot}}\xspace}			%  Msun
\newcommand{\lsun}{\ensuremath{L_{\odot}}\xspace}			%  Lsun
\newcommand{\lbol}{\ensuremath{L_{\mathrm{bol}}}}	%  Lbol
\newcommand{\ks}{K\ensuremath{_{\mathrm{s}}}}		%  Ks
\newcommand{\hh}{\ensuremath{\textrm{H}_{2}}\xspace}			%  H2
\newcommand{\formaldehyde}{\ensuremath{\textrm{H}_2\textrm{CO}}\xspace}
\newcommand{\formaldehydeIso}{\ensuremath{\textrm{H}_2~^{13}\textrm{CO}}\xspace}
\newcommand{\methanol}{\ensuremath{\textrm{CH}_3\textrm{OH}}\xspace}
\newcommand{\ortho}{\ensuremath{\textrm{o-H}_2\textrm{CO}}}
\newcommand{\oneone}{\ensuremath{1_{10}-1_{11}}\xspace}
\newcommand{\twotwo}{\ensuremath{2_{11}-2_{12}}\xspace}
\newcommand{\threethree}{\ensuremath{3_{12}-3_{13}}\xspace}
\newcommand{\threeohthree}{\ensuremath{3_{03}-2_{02}}\xspace}
\newcommand{\threetwotwo}{\ensuremath{3_{22}-2_{21}}\xspace}
\newcommand{\threetwoone}{\ensuremath{3_{21}-2_{20}}\xspace}
\newcommand{\JKaKc}{\ensuremath{J_{K_a K_c}}}
\newcommand{\water}{H$_{2}$O}		%  H2O
\newcommand{\feii}{\ion{Fe}{2}}		%  FeII
\newcommand{\kms}{\textrm{km~s}\ensuremath{^{-1}}\xspace}	%  km s-1
\newcommand{\sqcm}{cm$^{2}$\xspace}		%  cm^2
\newcommand{\percc}{\ensuremath{\textrm{cm}^{-3}}\xspace}
\newcommand{\persc}{\ensuremath{\textrm{cm}^{-2}}\xspace}
\newcommand{\persr}{\ensuremath{\textrm{sr}^{-1}}\xspace}
\newcommand{\peryr}{\ensuremath{\textrm{yr}^{-1}}\xspace}
\newcommand{\perkmspc}{\textrm{per~km~s}\ensuremath{^{-1}}\textrm{pc}\ensuremath{^{-1}}\xspace}	%  km s-1 pc-1
\newcommand{\um}{\ensuremath{\mu m}\xspace}    % micron
\newcommand{\mum}{$\mu$m}
\newcommand{\htwo}{\ensuremath{\textrm{H}_2}}    % micron
\newcommand{\Htwo}{\ensuremath{\textrm{H}_2}}    % micron
\newcommand{\HtwoO}{\ensuremath{\textrm{H}_2\textrm{O}}}    % micron
\newcommand{\htwoo}{\ensuremath{\textrm{H}_2\textrm{O}}}    % micron
\newcommand{\ha}{\ensuremath{\textrm{H}\alpha}}
\newcommand{\hb}{\ensuremath{\textrm{H}\beta}}
%\newcommand{\so}{ SO~(5~6)-(4~5) }
\newcommand{\regone}{Sh~2-201}
\newcommand{\regtwo}{AFGL~4029}
\newcommand{\regthree}{LW Cas Nebula}
\newcommand{\regfour}{IC 1848}
\newcommand{\regfive}{W5 NW}
\newcommand{\regsix}{SFO 11}
\newcommand{\so}{ SO~\ensuremath{5_6-4_5} }
\newcommand{\ammonia}{NH\ensuremath{_3}\xspace}
\newcommand{\region}{W5}
\newcommand{\twelveco}{\ensuremath{^{12}\textrm{CO}}}
\newcommand{\thirteenco}{\ensuremath{^{13}\textrm{CO}}}
\newcommand{\ceighteeno}{\ensuremath{\textrm{C}^{18}\textrm{O}}}
\def\ee#1{\ensuremath{\times10^{#1}}}
\newcommand{\degree}{\ensuremath{^{\circ}}}
\newcommand{\lowirac}{800}
\newcommand{\highirac}{8000}
\newcommand{\lowmips}{600}
\newcommand{\highmips}{5000}
\newcommand{\perbeam}{\ensuremath{\textrm{beam}^{-1}}}
\newcommand{\uchii}{UC\textsc{hii}\xspace}
\newcommand{\UCHII}{UC\textsc{hii}\xspace}
\newcommand{\hchii}{HC\textsc{hii}\xspace}
\newcommand{\hii}{H{\sc ii}\xspace}
\newcommand{\hi}{H{\sc i}\xspace}
\newcommand{\Hii}{H{\sc ii}\xspace}
\newcommand{\HII}{H{\sc ii}\xspace}
\renewcommand\arcdeg{\mbox{$^\circ$}\xspace} 
\renewcommand\arcmin{\mbox{$^\prime$}\xspace} 
\renewcommand\arcsec{\mbox{$^{\prime\prime}$}\xspace} 
\newcommand{\helv}{\fontfamily{phv}\selectfont}
\newcommand{\MUSTANG}{\textsc{MUSTANG-2}\xspace}
\newcommand{\MGPS}{\textsc{MGPS}\xspace}

%\newcommand{\arcmin}{'}

\def\Figure#1#2#3#4#5{
\begin{figure*}[htp]
\epsscale{#4}
\includegraphics[scale=#4,angle=#5]{#1}
\caption{#2}
\label{#3}
\end{figure*}
}

\def\SubFigure#1#2#3#4#5{
\begin{figure*}[htp]
\addtocounter{figure}{-1}
\epsscale{#4}
\includegraphics[angle=#5]{#1}
\caption{#2}
\label{#3}
\end{figure*}
}

\def\FigureTwo#1#2#3#4#5{
\begin{figure*}[htp]
\epsscale{#5}
\plottwo{#1}{#2}
\caption{#3}
\label{#4}
\end{figure*}
}

\def\SubFigureTwo#1#2#3#4#5{
\begin{figure*}[htp]
\addtocounter{figure}{-1}
\epsscale{#5}
\plottwo{#1}{#2}
\caption{#3}
\label{#4}
\end{figure*}
}

\def\FigureFour#1#2#3#4#5#6{
\begin{figure*}[htp]
\subfigure[]{ \includegraphics[width=3in]{#1} }
\subfigure[]{ \includegraphics[width=3in]{#2} }
\subfigure[]{ \includegraphics[width=3in]{#3} }
\subfigure[]{ \includegraphics[width=3in]{#4} }
\caption{#5}
\label{#6}
\end{figure*}
}

\def\OneColFigure#1#2#3#4#5{
\begin{figure}[htpb]
\epsscale{#4}
\includegraphics[scale=#4,angle=#5]{#1}
\caption{#2}
\label{#3}
\end{figure}
}


\def\Table#1#2#3#4#5#6{
\begin{deluxetable}{#1}
\tablewidth{0pt}
\tabletypesize{\footnotesize}
\tablecaption{#2}
\tablehead{#3}
\startdata
\label{#4}
#5
\enddata
#6
\end{deluxetable}
}

% Alter some LaTeX defaults for better treatment of figures:
    % See p.105 of "TeX Unbound" for suggested values.
    % See pp. 199-200 of Lamport's "LaTeX" book for details.
    %   General parameters, for ALL pages:
    \renewcommand{\topfraction}{0.9}	% max fraction of floats at top
    \renewcommand{\bottomfraction}{0.8}	% max fraction of floats at bottom
    %   Parameters for TEXT pages (not float pages):
    \setcounter{topnumber}{2}
    \setcounter{bottomnumber}{2}
    \setcounter{totalnumber}{4}     % 2 may work better
    \setcounter{dbltopnumber}{2}    % for 2-column pages
    \renewcommand{\dbltopfraction}{0.9}	% fit big float above 2-col. text
    \renewcommand{\textfraction}{0.07}	% allow minimal text w. figs
    %   Parameters for FLOAT pages (not text pages):
    \renewcommand{\floatpagefraction}{0.7}	% require fuller float pages
	% N.B.: floatpagefraction MUST be less than topfraction !!
    \renewcommand{\dblfloatpagefraction}{0.7}	% require fuller float pages


\setlength{\topmargin}{-0.5in}
\setlength{\textheight}{8.75in}
\setlength{\oddsidemargin}{-0.25in}
\setlength{\evensidemargin}{-0.25in}
\setlength{\textwidth}{7.0in}
\setlength{\parskip}{0.5mm}

% #1 - filename
% #2 - caption
% #3 - label
% #4 - epsscale
% #5 - R or L?
\def\WrapFigure#1#2#3#4#5#6{
%\begin{mdframed}
%\begin{minipage}{\textwidth}
\begin{wrapfigure}[#6]{#5}{0.45\textwidth}
%  \centercaption
%  \vspace{-14pt}
  \epsscale{#4}
  \includegraphics[scale=#4]{#1}
  \caption{#2}
  \label{#3}
\end{wrapfigure}
%\end{minipage}
%\end{mdframed}
}

%\input{macros}

\begin{document}
The MUSTANG Galactic Plane Survey (MGPS90) pilot

We report the results of a pilot program for a MUSTANG Galactic Plane survey

\section{Observations}

-description of observations, including date, tau, etc table

-description of data reduction.  We use the ninkasi mapper.


Effective Central Frequency Table:

\begin{tabular}{ll}
0.0 & 89.72 GHz \\
0.5 & 90.17 GHz \\
1.0 & 90.63 GHz \\
1.5 & 91.09 GHz \\
2.0 & 91.54 GHz \\
2.5 & 92.00 GHz \\
3.0 & 92.44 GHz \\
3.5 & 92.89 GHz \\
4.0 & 93.32 GHz 
\end{tabular}

\section{Catalogs}

We use \texttt{astrodendro} via the \texttt{dendrocat} wrapper to extract a
catalog of both compact structures.  In brief, \texttt{astrodendro} catalogs
hierarchically nested signal, effectively cataloging contoured regions.  For
the catalog described here, we included only the most compact structures, which
are the `leaves' in the catalog hierarchy.

To select primarily robust compact sources, we reject sources with a signal-to-noise
ratio less than 6.  We compute the noise for each source as the standard deviation
of pixel values in an aperture starting from 24\arcsec plus the source's major axis
size to 48\arcsec plus the source's major axis size.  This process was performed with
\texttt{dendrocat.autoreject}.

We experimented with a range of parameters for the noise threshold, the minimum
number of pixels, and the minimum $\Delta$ between levels.  While each of these
parameters significantly changes the total dendrogram structure, only the
threshold parameter had any effect on the retained leaves.

We then cross-match the resulting catalog with the Herschel HiGal
\citep{Elia2017a}, Spitzer GLIMPSE and MIPSGAL \citep{Churchwell2009a,Gutermuth2015a}, and MAGPIS 6cm and 20cm
\citep{Giveon2005a,Hoare2006a} and CORNISH 6cm \citep{Hoare2012a} catalogs.  In
W43, most (90\%) of the MUSTANG 3mm sources have a corresponding Herschel
compact source, 63\% have a centimeter association, and only 15\% have a
Spitzer association.  However, upon closer visual inspection, there are likely
infrared sources associated with the majority of the MUSTANG sources, but
many of these are extended objects that were not included in the Spitzer GLIMPSE
and MIPSGAL catalogs.

\subsection{W43 catalog}
We use the W43 region as our test area to examine the cataloging process and
cross-matching in detail.

While most MUSTANG sources are detected at either millimeter and submillimeter
wavelengths, representing dust emission, or at centimeter wavelengths,
representing free-free emission, there are some interesting exceptions.

Source 177 (Figure \ref{fig:sed177}) is detected at 3 mm, but has no counterpart from 160 to 1100 \um or
at centimeter wavelengths, and only a weak counterpart in the Herschel 70 \um
data. However, it has a clear and bright detection from 3.6 - 24 \um, indicating
the presence of a star with surrounding hot dust.  The lack of cm detection suggests
that hot dust is not part of an ionized region.  It is not clear why 3 mm emission
is detected.

Source 22 (Figure \ref{fig:sed22}) is detected at 20 cm, 1.1 and 3 mm, 70 \um,
and 3.6 - 24 \um.  The lack of far-infrared detections indicates that there is
no cold dust associated with the source, but the detection at 20 cm and
nondetection at 6 cm is not consistent.  The most likely explanation for this
discrepancy is that this source is variable and has a flickering \hii region.

One of the aims of this survey is to identify the youngest high-mass protostars.
Candidates are those with little to no infrared emission and very compact, optically
thick hypercompact \hii regions.

Source 83 (Figure \ref{fig:sed83}) is detected at 6 cm, 20 cm, and 3 mm, but has no compact source detections
at shorter wavelengths.  There is diffuse dust emission from 24 - 1100 \um around this
source, indicating that dense, cold dust is present.  This source is an excellent
candidate for a very young hypercompact \hii region.


\begin{figure*}[htp]
    \includegraphics[scale=1]{figures/G31_dend_contour_thr4_minn20_mind1_detection_histograms.pdf}
\caption{Histograms of the MUSTANG 3mm flux density.  In each panel, the colored region shows which
sources were detected in one or more of the Herschel Hi-Gal (top), Spitzer GLIMPSE or MIPSGAL (middle),
or MAGPIS 6 or 20 cm or CORNISH 6 cm (bottom) surveys.}
\label{fig:histogram}
\end{figure*}

\begin{figure*}[htp]
    \includegraphics[width=17cm]{figures/W43_catalog_overlay.pdf}
\caption{Image of the W43 region from MUSTANG.  The overplotted symbols show the locations
of Herschel-detected sources in cyan crosses, centimeter-detected sources in white x's, centimeter-detected Herschel nondetections as red circles,
and centimeter and Herschel non-detections in green.}
\label{fig:w43overview}
\end{figure*}

\begin{figure*}[htp]
    \includegraphics[width=17cm]{figures/SED_plot_G031_177.png}
\caption{SED plot for Source 177 from W43.}
\label{fig:sed177}
\end{figure*}

\begin{figure*}[htp]
    \includegraphics[width=17cm]{figures/SED_plot_G031_83.png}
\caption{SED plot for Source 83 from W43.}
\label{fig:sed83}
\end{figure*}

\begin{figure*}[htp]
    \includegraphics[width=17cm]{figures/SED_plot_G031_22.png}
\caption{SED plot for Source 22 from W43.}
\label{fig:sed22}
\end{figure*}

\end{document}
