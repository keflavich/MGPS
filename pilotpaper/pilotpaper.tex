\documentclass[twocolumn]{aastex62}
%Following line instructs TeXShop to use latex + ghostscript:
%!TEX TS-program = latex
%\documentclass[12pt]{article}
%\documentclass{/Users/adam/papers/latexfiles/emulateapj}
%\documentclass{/Users/adam/papers/latexfiles/emulateapj}
%\usepackage{/Users/adam/papers/latexfiles/emulateapj5}
%\usepackage{/Users/adam/papers/latexfiles/deluxetable}

%\special{! /pdfmark 
%            [/View [/XYZ null null 1]  % unspecified x and y offset, 100% zoom
%             /Page 1
%             /PageModehttps://www.sharelatex.com/4759396695vhbmwsvkbydp /UseThumbs % /UseNone /UserOutlines /UseThumbs /FullScreen
%            /DOCVIEW pdfmark 
%            }

\usepackage[utf8]{inputenc}
\usepackage{natbib}  % Requires natbib.sty, available from http://ads.harvard.edu/pubs/bibtex/astronat/
\usepackage{rotating}
\usepackage{savesym}
\savesymbol{singlespace}
\savesymbol{doublespace}
\usepackage{wrapfig}
\usepackage{setspace}
\usepackage{xspace}
\usepackage{color}
\usepackage{multicol}
\usepackage{mdframed}
%\citestyle{aa}  % (Author YYYY) references instead of (Author, YYYY)
%\bibliographystyle{/Users/adam/papers/latexfiles/apj_w_etal}
%\bibliographystyle{/Users/adam/papers/latexfiles/apj_w_etal_3auth}
\newcommand{\red}[1]{\textcolor{red}{#1}}
\newcommand{\todo}[1]{\textcolor{red}{#1}}
\usepackage{enumitem}
\setlist{nolistsep}
\setlist{noitemsep}
%\setlist{nosep}
%\setlist{nopartopsep}
%\setlist{noparbottomsep}
\usepackage[compact]{titlesec}
\titlespacing{\section}{0pt}{8pt}{0pt}


%%% this achieves the holy grail of 1 in margins all around!!! %%%
	%%%%%%%%%%%%%%%%%%%%%%%%%%%%%%
%	\oddsidemargin  0.0in
%	\evensidemargin 0.0in
%	\textwidth      7in
%	\headheight     0.0in
%	\topmargin      1.00in
%	\textheight=9in
	%%%%%%%%%%%%%%%%%%%%%%%%%%%%%%


% some macros that will probably be useful...
\newcommand{\paa}{Pa\ensuremath{\alpha}}
\newcommand{\brg}{Br\ensuremath{\gamma}}
\newcommand{\msun}{\ensuremath{M_{\odot}}\xspace}			%  Msun
\newcommand{\lsun}{\ensuremath{L_{\odot}}\xspace}			%  Lsun
\newcommand{\lbol}{\ensuremath{L_{\mathrm{bol}}}}	%  Lbol
\newcommand{\ks}{K\ensuremath{_{\mathrm{s}}}}		%  Ks
\newcommand{\hh}{\ensuremath{\textrm{H}_{2}}\xspace}			%  H2
\newcommand{\formaldehyde}{\ensuremath{\textrm{H}_2\textrm{CO}}\xspace}
\newcommand{\formaldehydeIso}{\ensuremath{\textrm{H}_2~^{13}\textrm{CO}}\xspace}
\newcommand{\methanol}{\ensuremath{\textrm{CH}_3\textrm{OH}}\xspace}
\newcommand{\ortho}{\ensuremath{\textrm{o-H}_2\textrm{CO}}}
\newcommand{\oneone}{\ensuremath{1_{10}-1_{11}}\xspace}
\newcommand{\twotwo}{\ensuremath{2_{11}-2_{12}}\xspace}
\newcommand{\threethree}{\ensuremath{3_{12}-3_{13}}\xspace}
\newcommand{\threeohthree}{\ensuremath{3_{03}-2_{02}}\xspace}
\newcommand{\threetwotwo}{\ensuremath{3_{22}-2_{21}}\xspace}
\newcommand{\threetwoone}{\ensuremath{3_{21}-2_{20}}\xspace}
\newcommand{\JKaKc}{\ensuremath{J_{K_a K_c}}}
\newcommand{\water}{H$_{2}$O}		%  H2O
\newcommand{\feii}{\ion{Fe}{2}}		%  FeII
\newcommand{\kms}{\textrm{km~s}\ensuremath{^{-1}}\xspace}	%  km s-1
\newcommand{\sqcm}{cm$^{2}$\xspace}		%  cm^2
\newcommand{\percc}{\ensuremath{\textrm{cm}^{-3}}\xspace}
\newcommand{\persc}{\ensuremath{\textrm{cm}^{-2}}\xspace}
\newcommand{\persr}{\ensuremath{\textrm{sr}^{-1}}\xspace}
\newcommand{\peryr}{\ensuremath{\textrm{yr}^{-1}}\xspace}
\newcommand{\perkmspc}{\textrm{per~km~s}\ensuremath{^{-1}}\textrm{pc}\ensuremath{^{-1}}\xspace}	%  km s-1 pc-1
\newcommand{\um}{\ensuremath{\mu m}\xspace}    % micron
\newcommand{\mum}{$\mu$m}
\newcommand{\htwo}{\ensuremath{\textrm{H}_2}}    % micron
\newcommand{\Htwo}{\ensuremath{\textrm{H}_2}}    % micron
\newcommand{\HtwoO}{\ensuremath{\textrm{H}_2\textrm{O}}}    % micron
\newcommand{\htwoo}{\ensuremath{\textrm{H}_2\textrm{O}}}    % micron
\newcommand{\ha}{\ensuremath{\textrm{H}\alpha}}
\newcommand{\hb}{\ensuremath{\textrm{H}\beta}}
%\newcommand{\so}{ SO~(5~6)-(4~5) }
\newcommand{\regone}{Sh~2-201}
\newcommand{\regtwo}{AFGL~4029}
\newcommand{\regthree}{LW Cas Nebula}
\newcommand{\regfour}{IC 1848}
\newcommand{\regfive}{W5 NW}
\newcommand{\regsix}{SFO 11}
\newcommand{\so}{ SO~\ensuremath{5_6-4_5} }
\newcommand{\ammonia}{NH\ensuremath{_3}\xspace}
\newcommand{\region}{W5}
\newcommand{\twelveco}{\ensuremath{^{12}\textrm{CO}}}
\newcommand{\thirteenco}{\ensuremath{^{13}\textrm{CO}}}
\newcommand{\ceighteeno}{\ensuremath{\textrm{C}^{18}\textrm{O}}}
\def\ee#1{\ensuremath{\times10^{#1}}}
\newcommand{\degree}{\ensuremath{^{\circ}}}
\newcommand{\lowirac}{800}
\newcommand{\highirac}{8000}
\newcommand{\lowmips}{600}
\newcommand{\highmips}{5000}
\newcommand{\perbeam}{\ensuremath{\textrm{beam}^{-1}}}
\newcommand{\uchii}{UC\textsc{hii}\xspace}
\newcommand{\UCHII}{UC\textsc{hii}\xspace}
\newcommand{\hchii}{HC\textsc{hii}\xspace}
\newcommand{\hii}{H{\sc ii}\xspace}
\newcommand{\hi}{H{\sc i}\xspace}
\newcommand{\Hii}{H{\sc ii}\xspace}
\newcommand{\HII}{H{\sc ii}\xspace}
\renewcommand\arcdeg{\mbox{$^\circ$}\xspace} 
\renewcommand\arcmin{\mbox{$^\prime$}\xspace} 
\renewcommand\arcsec{\mbox{$^{\prime\prime}$}\xspace} 
\newcommand{\helv}{\fontfamily{phv}\selectfont}
\newcommand{\MUSTANG}{\textsc{MUSTANG-2}\xspace}
\newcommand{\MGPS}{\textsc{MGPS}\xspace}

%\newcommand{\arcmin}{'}

\def\Figure#1#2#3#4#5{
\begin{figure*}[htp]
\epsscale{#4}
\includegraphics[scale=#4,angle=#5]{#1}
\caption{#2}
\label{#3}
\end{figure*}
}

\def\SubFigure#1#2#3#4#5{
\begin{figure*}[htp]
\addtocounter{figure}{-1}
\epsscale{#4}
\includegraphics[angle=#5]{#1}
\caption{#2}
\label{#3}
\end{figure*}
}

\def\FigureTwo#1#2#3#4#5{
\begin{figure*}[htp]
\epsscale{#5}
\plottwo{#1}{#2}
\caption{#3}
\label{#4}
\end{figure*}
}

\def\SubFigureTwo#1#2#3#4#5{
\begin{figure*}[htp]
\addtocounter{figure}{-1}
\epsscale{#5}
\plottwo{#1}{#2}
\caption{#3}
\label{#4}
\end{figure*}
}

\def\FigureFour#1#2#3#4#5#6{
\begin{figure*}[htp]
\subfigure[]{ \includegraphics[width=3in]{#1} }
\subfigure[]{ \includegraphics[width=3in]{#2} }
\subfigure[]{ \includegraphics[width=3in]{#3} }
\subfigure[]{ \includegraphics[width=3in]{#4} }
\caption{#5}
\label{#6}
\end{figure*}
}

\def\OneColFigure#1#2#3#4#5{
\begin{figure}[htpb]
\epsscale{#4}
\includegraphics[scale=#4,angle=#5]{#1}
\caption{#2}
\label{#3}
\end{figure}
}


\def\Table#1#2#3#4#5#6{
\begin{deluxetable}{#1}
\tablewidth{0pt}
\tabletypesize{\footnotesize}
\tablecaption{#2}
\tablehead{#3}
\startdata
\label{#4}
#5
\enddata
#6
\end{deluxetable}
}

% Alter some LaTeX defaults for better treatment of figures:
    % See p.105 of "TeX Unbound" for suggested values.
    % See pp. 199-200 of Lamport's "LaTeX" book for details.
    %   General parameters, for ALL pages:
    \renewcommand{\topfraction}{0.9}	% max fraction of floats at top
    \renewcommand{\bottomfraction}{0.8}	% max fraction of floats at bottom
    %   Parameters for TEXT pages (not float pages):
    \setcounter{topnumber}{2}
    \setcounter{bottomnumber}{2}
    \setcounter{totalnumber}{4}     % 2 may work better
    \setcounter{dbltopnumber}{2}    % for 2-column pages
    \renewcommand{\dbltopfraction}{0.9}	% fit big float above 2-col. text
    \renewcommand{\textfraction}{0.07}	% allow minimal text w. figs
    %   Parameters for FLOAT pages (not text pages):
    \renewcommand{\floatpagefraction}{0.7}	% require fuller float pages
	% N.B.: floatpagefraction MUST be less than topfraction !!
    \renewcommand{\dblfloatpagefraction}{0.7}	% require fuller float pages


\setlength{\topmargin}{-0.5in}
\setlength{\textheight}{8.75in}
\setlength{\oddsidemargin}{-0.25in}
\setlength{\evensidemargin}{-0.25in}
\setlength{\textwidth}{7.0in}
\setlength{\parskip}{0.5mm}

% #1 - filename
% #2 - caption
% #3 - label
% #4 - epsscale
% #5 - R or L?
\def\WrapFigure#1#2#3#4#5#6{
%\begin{mdframed}
%\begin{minipage}{\textwidth}
\begin{wrapfigure}[#6]{#5}{0.45\textwidth}
%  \centercaption
%  \vspace{-14pt}
  \epsscale{#4}
  \includegraphics[scale=#4]{#1}
  \caption{#2}
  \label{#3}
\end{wrapfigure}
%\end{minipage}
%\end{mdframed}
}

%\input{macros}

\def\todo#1{{\textcolor{red}{TODO: #1}}}

\begin{document}
The MUSTANG Galactic Plane Survey (MGPS90) pilot

We report the results of a pilot program for a MUSTANG Galactic Plane survey

\section{Observations}

A summary of the reported observations is given in Table \ref{tab:observations}.


\begin{table*}[htp]
\centering
%\begin{minipage}{130mm}
\caption{Observation Summary}
\begin{tabular}{lll}
    \label{tab:observations}
Target   &      Time (hr) &       Sessions   \\
\hline
W51      &      1.02      &        01        \\
W49      &      1.02      &        01, 02    \\
SgrB2    &      1.35      &   02, 03, 04, 05 \\
GAL031   &      1.49      &        02, 03    \\
W33      &      1.02      &        03        \\
GAL029   &      1.30      &        04,05     \\
GAL034   &     $\sim0.5$  &   05             \\
\hline
\end{tabular}
\par
\todo{Clean up formatting, add dates of sessions}
The tabulated times are those in the maps (just in the scans that were used to
make a given map).  Notes:
In GAL034, only 6 of the constant-latitude scans were completed, so only the bottom
1/3 of map has full cross linking
\end{table*}

\subsection{Calibration}

The calibrations were done according to Charles' calibration scripts. Known
point sources are observed at regular intervals each night.

The calibration procedure is, summarily, as follows:
\begin{enumerate}
    \item A calibration for the detector array is found using a skydip and the
        opacity (at 90 GHz) as given by CLEO (Control Library for Operators and
        Engineers; \url{http://www.gb.nrao.edu/~rmaddale/CLEOManual/}), to get
        each timestream into
        antenna temperature.
    \item A map is made (in azimuth/elevation coordinates) of each calibration
        (point source) scan in IDL.
        \begin{enumerate}
            \item A single 2-D Gaussian is fit to the point source. The centroid and widths are recorded.
            \item Fixing the centroid as found above, a double Gaussian (two 1-D Gaussians) is fit.
            \item The beam volume is calculated from the map within a 60"
                aperture as well as analytically from the double Gaussian. The
                adopted beam volume is often that from the double Gaussian, but
                if the fits are poor, the 60" aperture value is generally
                adopted. If both values are questionable, or some other
                indicator of reliability fails, the beam volume of that
                individual scan will be overlooked.
        \end{enumerate}
    \item
        \begin{enumerate}
            \item The peaks of secondary calibrators are normalized by the mean
                for each specific secondary calibrator. These peaks are tied
                into a primary calibrator, whose peak is scaled to the expected
                peak, in Jy(/beam), which is determined from planetary models
                if a planet, or from interpolation of ALMA data, if only an
                ALMA calibrator was accessible in a given night. The scaling is
                linearly interpolated between calibration scans.
            \item Conversion to Kelvin accounts for the beam volume. As such,
                the beam volumes are interpolated between scans.
        \end{enumerate}
    \item Calibration to Jy, conversion from Jy/beam to Kelvin, opacities, and
        pointing offsets are all recorded in an IDL save file, which is read in
        and applied to the processing of the science scans.
\end{enumerate}

The absolute accuracy of these calibrations should be reasonable and about 10\%.
Some of this uncertainty is from the extrapolation in time and frequency of the ALMA
sources (the ALMA band is different from MUSTANG but there are measurements at
$\sim~100$ and 91 GHz), some is the error in the point source fluxes from ALMA and
some is from our knowledge of $\tau$ (the sky optical depth) during the observations.

\subsection{Map Making}
Maps were made using the MINKASI algorithm \citep{}\todo{Simon: could you fill in a citation here?}.
We used smoothed power spectra from a singular value decomposition (SVD) of the data on a scan by scan basis to
obtain a noise model.
This model has does not work well if there are strong sources,
so we added features to work around this by subdividing timestreams and taking
the lowest power spectra from each scan.  We also included a fudge factor for
lowering the power spectra in the signal band.

We followed an iterative process to obtain the best maps.  A map is made, the
result then clipped at some level above any artifacts in that iteration and the
results subtracted from the timestreams.  In each loop, the clipping level was
reduced and the noise model recalculated.  In the last loops (in which all
strong signal should have been removed) the full SVD noise model could be used
(which tended to give better results on faint features. For W33, three
iterations produced optimal results; the other regions required more
iterations.

For some fields, notably G34, we only obtained scans in one direction.  Future
observations filling in the orthogonal scan direction will be needed to eliminate
the resulting scan-direction striping features.

The map making process assumes the mean incoming intensity is zero.  This assumption
encodes a large angular scale filter such that angular scales larger than $\sim4.25$\arcmin
are not present in the data.  This filtering is visible as negative bowls in the images,
especially in the Sgr B2 / Galactic Center field.

\subsection{Sensitivity and beam size}

The noise in the maps is given per 1" pixel - given there are 80 (Gaussian
beam) to 140 (variance beam) arcsecond sq per beam one can expect the noise in
the unsmoothed maps to drop by an order of magnitude when proper smoothing is
applied.

\begin{table*}[htp]
    \begin{tabular}{llll}
TARGET   &    IDL   &   MINKASI & MINKASI 4\arcsec smooth\\
\hline
GAL029   &      3.7 &     2.6   & 1.1\\
GAL031   &      6.1 &     4.1   & 1.8\\
SgrB2    &     8.0  &    7.4    & 2.2\\
W33      &   8.0    &  2.6      & 1.7\\
W49      &   6.8    &  3.7      & 1.8\\
W51      &   4.3    &  4.9      & 1.1\\
\hline
    \end{tabular}
These noise measurements are estimates based on relatively signal-free regions of the maps.
\todo{Decide whether to keep the left two columns or just report the MINKASI noise}
\end{table*}

The effective beam size in the delivered maps is the convolution of the
intrinsic FWHM = 8.1\arcsec beam with a 4\arcsec Gaussian kernel, resulting in
a 9\arcsec beam.


\subsection{Pointing Accuracy}
Several corrections to the raw timestream data were required to produce maps.  Individual scans were noted
to have point sources shifted by up to half a beam ($\sim4\arcsec$), indicating a timing error between
the MUSTANG-2 pointing data and the true telescope pointing.  To ensure that point sources were coincident
in the maps, scans were cross-correlated with a first-iteration map, then assigned a new timing offset.
The timing errors ranged from $\sim10$ to $30$ milliseconds.

Additional half-beam pointing errors were noted in some individual scans, resulting in additional streaking
artifacts in the data.  Most of these issues disappeared after smoothing the data with the 4\arcsec kernel.

Because several corrections were required, we assume the absolute pointing
accuracy is $\sim4\arcsec$.  We therefore cross-correlated the MUSTANG-2 maps
with \todo{BGPS, ATLASGAL, Planck, HiGal are all fine candidates to cross-correlate
against and measure the pointing offset.  We probably want to deliver something consistent with one or
more of these surveys}

\subsection{Effective Central Frequency}
The MUSTANG-2 bandpass filter is approximately flat over the range XX to YY GHz.
We convolved the bandpass filter with power law flux density distributions with $S_\nu\propto\nu^{\alpha}$
to obtain the true effective central frequency of the bandpass for these assumed
continuous distributions.  They are reported in Table \ref{tab:centralfreq}.

\begin{table*}[htp]
\begin{tabular}{ll}
    \label{tab:centralfreq}
$\alpha$ & Frequency \\
\hline
0.0 & 89.72 GHz \\
0.5 & 90.17 GHz \\
1.0 & 90.63 GHz \\
1.5 & 91.09 GHz \\
2.0 & 91.54 GHz \\
2.5 & 92.00 GHz \\
3.0 & 92.44 GHz \\
3.5 & 92.89 GHz \\
4.0 & 93.32 GHz \\
\hline
\end{tabular}
\end{table*}

\section{Catalogs}

We use \texttt{astrodendro} via the \texttt{dendrocat} wrapper to extract a
catalog of both compact structures.  In brief, \texttt{astrodendro} catalogs
hierarchically nested signal, effectively cataloging contoured regions.  For
the catalog described here, we included only the most compact structures, which
are the `leaves' in the catalog hierarchy.

To select primarily robust compact sources, we reject sources with a signal-to-noise
ratio less than 6.  We compute the noise for each source as the standard deviation
of pixel values in an aperture starting from 24\arcsec plus the source's major axis
size to 48\arcsec plus the source's major axis size.  This process was performed with
\texttt{dendrocat.autoreject}.

We experimented with a range of parameters for the noise threshold, the minimum
number of pixels, and the minimum $\Delta$ between levels.  While each of these
parameters significantly changes the total dendrogram structure, only the
threshold parameter had any effect on the retained leaves.

\subsection{Catalog cross-matching}
\label{sec:catalogmatching}
We then cross-match the resulting catalog with the Herschel HiGal
\citep{Elia2017a}, Spitzer GLIMPSE and MIPSGAL
\citep{Churchwell2009a,Gutermuth2015a}, and MAGPIS 6cm and 20cm
\citep{Giveon2005a,Hoare2006a} and CORNISH 6cm \citep{Hoare2012a} catalogs.
Matches in these catalogs are included if there is a source within 10\arcsec
(approximately the MUSTANG beam FWHM) of the MGPS catalog entry.

\subsection{W43 catalog}
We use the W43 region as our test area to examine the cataloging process and
cross-matching in detail.  Cataloging was performed on the file
\texttt{GAL\_031\_precon\_2\_arcsec\_pass\_9.fits}.

In W43, most (80\%) of the 108 MUSTANG 3mm sources have a corresponding Herschel
compact source, 49\% have a centimeter association, and only 19\% have a
Spitzer association.  However, upon closer visual inspection, there are likely
infrared sources associated with the majority of the MUSTANG sources, but many
of these are extended objects that were not included in the Spitzer GLIMPSE and
MIPSGAL catalogs.

The breakdown of match fraction as a function of flux density is shown in histograms
in Figure \ref{fig:histogram}.

\subsubsection{Individual sources}
While most MUSTANG sources are detected at either millimeter and submillimeter
wavelengths, representing dust emission, or at centimeter wavelengths,
representing free-free emission, there are some interesting exceptions.

Sources with no counterparts at other wavelengths: \\
G29.955-0.110 (faint) \\
G29.973-0.285 (Herschel not available) \\
G30.159-0.466 (faint) \\
G30.008-0.039 (faint/questionable) \\
G30.721-0.039 (diffuse counterparts?) \\
G29.969+0.033 (faint) \\
G30.364+0.027 (diffuse, faint)\\
G31.371+0.021 (faint) \\


Sources with 3mm and cm-and-longer detections, but no IR or FIR or submm detections:\\
G30.437-0.206 \\
G31.150-0.189 \\
G31.387-0.383 \\
G30.330+0.090 \\
G31.440+0.510 (no Herschel 70 \um or 160 \um data available) \\


Sources with 3mm and Spitzer detections, but no FIR or cm:
G30.248-0.192 \\

IRDCs:\\
G30.400-0.296 \\
G30.419-0.231 \\
G30.689+0.032 (faint, questionable)\\
G30.920+0.091 \\
G31.382+0.318 \\
G31.581+0.077 \\



% Source G30.651-0.060 (Figure \ref{fig:sed177}) is detected at 3 mm, but has no counterpart from 160 to 1100 \um or
% at centimeter wavelengths, and only a weak counterpart in the Herschel 70 \um
% data. However, it has a clear and bright detection from 3.6 - 24 \um, indicating
% the presence of a star with surrounding hot dust.  The lack of cm detection suggests
% that hot dust is not part of an ionized region.  It is not clear why 3 mm emission
% is detected.

% Source G29.979+0.021 (Figure \ref{fig:sed22}) is detected at 20 cm, 1.1 and 3 mm, 70 \um,
% and 3.6 - 24 \um.  The lack of far-infrared detections indicates that there is
% no cold dust associated with the source, but the detection at 20 cm and
% nondetection at 6 cm is not consistent.  The most likely explanation for this
% discrepancy is that this source is variable and has a flickering \hii region.

The brightest source with a Herschel nondetection is G30.711-0.019.  It is in a
region filled with extended emission at many wavelengths.  The 3 mm emission
closely matches the 20 cm emission and 70 \um and shorter-wavelength emission,
but does not match the longer-wavelength millimeter emission, suggesting that
the 3 mm is dominated by free-free emission.  The millimeter and centimeter
emission are offset from one another, which is why there was no catalog source
match; the free-free emission wraps around the dust emission.  This source is
therefore just a particularly bright patch of diffuse free-free emission.

One of the aims of this survey is to identify the youngest high-mass
protostars.  Candidates are those with little to no infrared emission and very
compact, optically thick hypercompact \hii regions.

Hypercompact \hii regions are defined as $\lesssim0.03$ pc ($\lesssim6000$ AU)
\hii regions around young stars \citep{Kurtz2005b}.  They have high ionized
gas density, resulting in a spectrum that remains optically thick to relatively
high frequency.  For a star to produce an \hii region so compact, it must be
embedded in a high-density medium.  We therefore select candidate HC\hii regions
as those fitting these criteria:
\begin{enumerate}
    \item $S_{3 \mathrm{mm}} > S_{6 \mathrm{cm}}$ or $S_{3 \mathrm{mm}} > S_{20 \mathrm{cm}}$,
        or a nondetection at 6 and 20 cm.  This requirement selects free-free sources that are optically
        thick or moderately optically thick at long wavelengths.
    \item $S_{3 \mathrm{mm}} / S_{1.1 \mathrm{mm}} < \left(S_{3
        \mathrm{mm}} / S_{1.1 \mathrm{mm}})^{-3}\right)$,  where $\alpha=-3$ is the
        spectral index for optically thin dust with an opacity index $\beta=1$.
        This requirement selects dust-detected sources in which there is some
        indication of an excess of free-free emission over pure dust emission.
\end{enumerate}

G29.977-0.047: SED is dust-only, but there is excess at 3 mm plus a bright 70 \um source.  Could be?
G30.009-0.273: Compact HII region?  6cm detection, 20cm limit
G30.099+0.077: Candidate?  24, 70 \um bright.  CH3OH, H2O source

General conclusion: these are all good candidates, but they may contain mixes of HII regions and MYSOs.


Source G30.437-0.206 (Figure \ref{fig:sed83}) is detected at 6 cm, 20 cm, and 3
mm, but has no compact source detections at shorter wavelengths.  There is
diffuse dust emission from 24 - 1100 \um around this source, indicating that
dense, cold dust is present.  This source is an excellent candidate for a very
young hypercompact \hii region.


\begin{figure*}[htp]
    \includegraphics[scale=1]{figures/G31_dend_contour_thr4_minn20_mind1_detection_histograms.pdf}
\caption{Histograms of the MUSTANG 3mm flux density.  In each panel, the colored region shows which
sources were detected in one or more of the Herschel Hi-Gal (top), Spitzer GLIMPSE or MIPSGAL (middle),
or MAGPIS 6 or 20 cm or CORNISH 6 cm (bottom) surveys.}
\label{fig:histogram}
\end{figure*}

\begin{figure*}[htp]
    \includegraphics[width=17cm]{figures/W43_catalog_overlay.pdf}
\caption{Image of the W43 region from MUSTANG.  The overplotted symbols show the locations
of Herschel-detected sources in cyan crosses, centimeter-detected sources in white x's, centimeter-detected Herschel nondetections as red circles,
and centimeter and Herschel non-detections in green.}
\label{fig:w43overview}
\end{figure*}

% no longer exists
% \begin{figure*}[htp]
%     \includegraphics[width=17cm]{figures/SED_plot_G031_177.png}
% \caption{SED plot for Source  G30.651-0.060 from W43.  The panels are labeled with the
% survey and frequency used for the plot.  The data come from the MAGPIS cutout
% server (\url{https://third.ucllnl.org/gps/}) and the Herschel Hi-Gal image
% server (\url{https://tools.ssdc.asi.it/HiGAL.jsp}); references are given in the
% text.  The bottom-right panel shows an SED constructed from the sources matched
% from the catalogs described in Section \ref{sec:catalogmatching}.
% This source is detected only at 3 mm and in the near infrared; it is not clear
% what class of object it is.
% }
% \label{fig:sed177}
% \end{figure*}

% no longer exists
% \begin{figure*}[htp]
%     \includegraphics[width=17cm]{figures/SED_plot_G031_22.png}
% \caption{SED plot for Source G29.979+0.021 from W43.  See Figure \ref{fig:sed177} for a description.
% Because this source is detected at 20 cm and 3 mm, but not at the intermediate 6 cm wavelength,
% it is likely a variable \hii region associated with the central massive star.
% }
% \label{fig:sed22}
% \end{figure*}


\begin{figure*}[htp]
    \includegraphics[width=17cm]{figures/SED_plot_G30.437-0.206.png}
    \caption{SED plot for Source G30.437-0.206 from W43.  The panels are
    labeled with the survey and frequency used for the plot.  The data come
    from the MAGPIS cutout server (\url{https://third.ucllnl.org/gps/}) and the
    Herschel Hi-Gal image server (\url{https://tools.ssdc.asi.it/HiGAL.jsp});
    references are given in the text.  The bottom-right panel shows an SED
    constructed from the sources matched from the catalogs described in Section
    \ref{sec:catalogmatching}.  This source is detected only at 3 mm and longer
    wavelengths.  It is likely a hypercompact \hii region.}
\label{fig:sed83}
\end{figure*}

\input{solobib.tex}
\end{document}
