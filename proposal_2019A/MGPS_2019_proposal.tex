%Following line instructs TeXShop to use latex + ghostscript:
%!TEX TS-program = latex
%\documentclass[12pt]{article}
%\documentclass{/Users/adam/papers/latexfiles/emulateapj}
%\documentclass{/Users/adam/papers/latexfiles/emulateapj}
%\usepackage{/Users/adam/papers/latexfiles/emulateapj5}
%\usepackage{/Users/adam/papers/latexfiles/deluxetable}

%\special{! /pdfmark 
%            [/View [/XYZ null null 1]  % unspecified x and y offset, 100% zoom
%             /Page 1
%             /PageModehttps://www.sharelatex.com/4759396695vhbmwsvkbydp /UseThumbs % /UseNone /UserOutlines /UseThumbs /FullScreen
%            /DOCVIEW pdfmark 
%            }

\usepackage[utf8]{inputenc}
\usepackage{natbib}  % Requires natbib.sty, available from http://ads.harvard.edu/pubs/bibtex/astronat/
\usepackage{rotating}
\usepackage{savesym}
\savesymbol{singlespace}
\savesymbol{doublespace}
\usepackage{wrapfig}
\usepackage{setspace}
\usepackage{xspace}
\usepackage{color}
\usepackage{multicol}
\usepackage{mdframed}
%\citestyle{aa}  % (Author YYYY) references instead of (Author, YYYY)
%\bibliographystyle{/Users/adam/papers/latexfiles/apj_w_etal}
%\bibliographystyle{/Users/adam/papers/latexfiles/apj_w_etal_3auth}
\newcommand{\red}[1]{\textcolor{red}{#1}}
\newcommand{\todo}[1]{\textcolor{red}{#1}}
\usepackage{enumitem}
\setlist{nolistsep}
\setlist{noitemsep}
%\setlist{nosep}
%\setlist{nopartopsep}
%\setlist{noparbottomsep}
\usepackage[compact]{titlesec}
\titlespacing{\section}{0pt}{8pt}{0pt}


%%% this achieves the holy grail of 1 in margins all around!!! %%%
	%%%%%%%%%%%%%%%%%%%%%%%%%%%%%%
%	\oddsidemargin  0.0in
%	\evensidemargin 0.0in
%	\textwidth      7in
%	\headheight     0.0in
%	\topmargin      1.00in
%	\textheight=9in
	%%%%%%%%%%%%%%%%%%%%%%%%%%%%%%


% some macros that will probably be useful...
\newcommand{\paa}{Pa\ensuremath{\alpha}}
\newcommand{\brg}{Br\ensuremath{\gamma}}
\newcommand{\msun}{\ensuremath{M_{\odot}}\xspace}			%  Msun
\newcommand{\lsun}{\ensuremath{L_{\odot}}\xspace}			%  Lsun
\newcommand{\lbol}{\ensuremath{L_{\mathrm{bol}}}}	%  Lbol
\newcommand{\ks}{K\ensuremath{_{\mathrm{s}}}}		%  Ks
\newcommand{\hh}{\ensuremath{\textrm{H}_{2}}\xspace}			%  H2
\newcommand{\formaldehyde}{\ensuremath{\textrm{H}_2\textrm{CO}}\xspace}
\newcommand{\formaldehydeIso}{\ensuremath{\textrm{H}_2~^{13}\textrm{CO}}\xspace}
\newcommand{\methanol}{\ensuremath{\textrm{CH}_3\textrm{OH}}\xspace}
\newcommand{\ortho}{\ensuremath{\textrm{o-H}_2\textrm{CO}}}
\newcommand{\oneone}{\ensuremath{1_{10}-1_{11}}\xspace}
\newcommand{\twotwo}{\ensuremath{2_{11}-2_{12}}\xspace}
\newcommand{\threethree}{\ensuremath{3_{12}-3_{13}}\xspace}
\newcommand{\threeohthree}{\ensuremath{3_{03}-2_{02}}\xspace}
\newcommand{\threetwotwo}{\ensuremath{3_{22}-2_{21}}\xspace}
\newcommand{\threetwoone}{\ensuremath{3_{21}-2_{20}}\xspace}
\newcommand{\JKaKc}{\ensuremath{J_{K_a K_c}}}
\newcommand{\water}{H$_{2}$O}		%  H2O
\newcommand{\feii}{\ion{Fe}{2}}		%  FeII
\newcommand{\kms}{\textrm{km~s}\ensuremath{^{-1}}\xspace}	%  km s-1
\newcommand{\sqcm}{cm$^{2}$\xspace}		%  cm^2
\newcommand{\percc}{\ensuremath{\textrm{cm}^{-3}}\xspace}
\newcommand{\persc}{\ensuremath{\textrm{cm}^{-2}}\xspace}
\newcommand{\persr}{\ensuremath{\textrm{sr}^{-1}}\xspace}
\newcommand{\peryr}{\ensuremath{\textrm{yr}^{-1}}\xspace}
\newcommand{\perkmspc}{\textrm{per~km~s}\ensuremath{^{-1}}\textrm{pc}\ensuremath{^{-1}}\xspace}	%  km s-1 pc-1
\newcommand{\um}{\ensuremath{\mu m}\xspace}    % micron
\newcommand{\mum}{$\mu$m}
\newcommand{\htwo}{\ensuremath{\textrm{H}_2}}    % micron
\newcommand{\Htwo}{\ensuremath{\textrm{H}_2}}    % micron
\newcommand{\HtwoO}{\ensuremath{\textrm{H}_2\textrm{O}}}    % micron
\newcommand{\htwoo}{\ensuremath{\textrm{H}_2\textrm{O}}}    % micron
\newcommand{\ha}{\ensuremath{\textrm{H}\alpha}}
\newcommand{\hb}{\ensuremath{\textrm{H}\beta}}
%\newcommand{\so}{ SO~(5~6)-(4~5) }
\newcommand{\regone}{Sh~2-201}
\newcommand{\regtwo}{AFGL~4029}
\newcommand{\regthree}{LW Cas Nebula}
\newcommand{\regfour}{IC 1848}
\newcommand{\regfive}{W5 NW}
\newcommand{\regsix}{SFO 11}
\newcommand{\so}{ SO~\ensuremath{5_6-4_5} }
\newcommand{\ammonia}{NH\ensuremath{_3}\xspace}
\newcommand{\region}{W5}
\newcommand{\twelveco}{\ensuremath{^{12}\textrm{CO}}}
\newcommand{\thirteenco}{\ensuremath{^{13}\textrm{CO}}}
\newcommand{\ceighteeno}{\ensuremath{\textrm{C}^{18}\textrm{O}}}
\def\ee#1{\ensuremath{\times10^{#1}}}
\newcommand{\degree}{\ensuremath{^{\circ}}}
\newcommand{\lowirac}{800}
\newcommand{\highirac}{8000}
\newcommand{\lowmips}{600}
\newcommand{\highmips}{5000}
\newcommand{\perbeam}{\ensuremath{\textrm{beam}^{-1}}}
\newcommand{\uchii}{UC\textsc{hii}\xspace}
\newcommand{\UCHII}{UC\textsc{hii}\xspace}
\newcommand{\hchii}{HC\textsc{hii}\xspace}
\newcommand{\hii}{H{\sc ii}\xspace}
\newcommand{\hi}{H{\sc i}\xspace}
\newcommand{\Hii}{H{\sc ii}\xspace}
\newcommand{\HII}{H{\sc ii}\xspace}
\renewcommand\arcdeg{\mbox{$^\circ$}\xspace} 
\renewcommand\arcmin{\mbox{$^\prime$}\xspace} 
\renewcommand\arcsec{\mbox{$^{\prime\prime}$}\xspace} 
\newcommand{\helv}{\fontfamily{phv}\selectfont}
\newcommand{\MUSTANG}{\textsc{MUSTANG-2}\xspace}
\newcommand{\MGPS}{\textsc{MGPS}\xspace}

%\newcommand{\arcmin}{'}

\def\Figure#1#2#3#4#5{
\begin{figure*}[htp]
\epsscale{#4}
\includegraphics[scale=#4,angle=#5]{#1}
\caption{#2}
\label{#3}
\end{figure*}
}

\def\SubFigure#1#2#3#4#5{
\begin{figure*}[htp]
\addtocounter{figure}{-1}
\epsscale{#4}
\includegraphics[angle=#5]{#1}
\caption{#2}
\label{#3}
\end{figure*}
}

\def\FigureTwo#1#2#3#4#5{
\begin{figure*}[htp]
\epsscale{#5}
\plottwo{#1}{#2}
\caption{#3}
\label{#4}
\end{figure*}
}

\def\SubFigureTwo#1#2#3#4#5{
\begin{figure*}[htp]
\addtocounter{figure}{-1}
\epsscale{#5}
\plottwo{#1}{#2}
\caption{#3}
\label{#4}
\end{figure*}
}

\def\FigureFour#1#2#3#4#5#6{
\begin{figure*}[htp]
\subfigure[]{ \includegraphics[width=3in]{#1} }
\subfigure[]{ \includegraphics[width=3in]{#2} }
\subfigure[]{ \includegraphics[width=3in]{#3} }
\subfigure[]{ \includegraphics[width=3in]{#4} }
\caption{#5}
\label{#6}
\end{figure*}
}

\def\OneColFigure#1#2#3#4#5{
\begin{figure}[htpb]
\epsscale{#4}
\includegraphics[scale=#4,angle=#5]{#1}
\caption{#2}
\label{#3}
\end{figure}
}


\def\Table#1#2#3#4#5#6{
\begin{deluxetable}{#1}
\tablewidth{0pt}
\tabletypesize{\footnotesize}
\tablecaption{#2}
\tablehead{#3}
\startdata
\label{#4}
#5
\enddata
#6
\end{deluxetable}
}

% Alter some LaTeX defaults for better treatment of figures:
    % See p.105 of "TeX Unbound" for suggested values.
    % See pp. 199-200 of Lamport's "LaTeX" book for details.
    %   General parameters, for ALL pages:
    \renewcommand{\topfraction}{0.9}	% max fraction of floats at top
    \renewcommand{\bottomfraction}{0.8}	% max fraction of floats at bottom
    %   Parameters for TEXT pages (not float pages):
    \setcounter{topnumber}{2}
    \setcounter{bottomnumber}{2}
    \setcounter{totalnumber}{4}     % 2 may work better
    \setcounter{dbltopnumber}{2}    % for 2-column pages
    \renewcommand{\dbltopfraction}{0.9}	% fit big float above 2-col. text
    \renewcommand{\textfraction}{0.07}	% allow minimal text w. figs
    %   Parameters for FLOAT pages (not text pages):
    \renewcommand{\floatpagefraction}{0.7}	% require fuller float pages
	% N.B.: floatpagefraction MUST be less than topfraction !!
    \renewcommand{\dblfloatpagefraction}{0.7}	% require fuller float pages


\setlength{\topmargin}{-0.5in}
\setlength{\textheight}{8.75in}
\setlength{\oddsidemargin}{-0.25in}
\setlength{\evensidemargin}{-0.25in}
\setlength{\textwidth}{7.0in}
\setlength{\parskip}{0.5mm}

% #1 - filename
% #2 - caption
% #3 - label
% #4 - epsscale
% #5 - R or L?
\def\WrapFigure#1#2#3#4#5#6{
%\begin{mdframed}
%\begin{minipage}{\textwidth}
\begin{wrapfigure}[#6]{#5}{0.45\textwidth}
%  \centercaption
%  \vspace{-14pt}
  \epsscale{#4}
  \includegraphics[scale=#4]{#1}
  \caption{#2}
  \label{#3}
\end{wrapfigure}
%\end{minipage}
%\end{mdframed}
}

\include{bib_macros}
\begin{document}
%{\color{red}anyone know how to change the title (not of the tex, but of the webpage?)}
%{\color{blue}Clicking the inverse triangle in the dash board page and then click "rename".\color{black} {\color{red} Thanks!}

%{\color{red}(Adam's note)   If you'd like to add your comments, please pick a color and use the \textbackslash
%color tag as has been done here.}
%{\color{green}(Charles' comments)}

A Galactic Plane Survey (GPS) at 3 mm and 8\arcsec resolution will provide a
unique legacy data set that no other telescope, existing or planned, will be
able to achieve.  With \MUSTANG on the GBT, a full GPS is now practical.  This
proposal is to complete the first segment of a Galactic plane survey.  With
our previous pilot program, we demonstrated the power and efficiency of \MUSTANG
to survey the plane, and we demonstrated combination with zero-spacing provided
from the Planck mission and high-resolution observations from ALMA.


\underline{\bf Background:}
Galactic plane surveys provide the critical path to understanding the Milky Way
as a Galaxy.  They provide us with the ability to perform complete, unbiased
censuses and therefore the tools to study populations of forming stars.

A 3 mm continuum survey with MUSTANG will address the following key questions:

\begin{enumerate}

    \item Where are the youngest high-mass protostars in the Galaxy, and how
        long does the early \hchii stage last?
    \item What is the structure of the dust spectral index?  Do grains grow
        during the collapse of molecular clouds, or only during the protostellar
        core phase?
    \item Where are the spectral turnover points of \hii regions?  What is the
        density structure, and therefore the photoevaporation rate, of 
        molecular clouds around embedded giant \hii regions?

\end{enumerate}


% Massive stellar clusters dominate the energy budget and observed light in star-forming
% galaxies.  While massive stars may have a formation mode comparable to their
% low-mass counterparts, most massive stars form in dense clusters.  In our own
% galaxy, the most massive $\sim$20 clusters produce half of the total ionizing
% radiation \citep{Murray2010a}: massive clusters hold the key to
% understanding feedback on Galactic scales.  
% %Given the higher gas densities in
% %all galaxies in the early universe, massive clusters represent the class of
% %regions in which most stars formed, including our own solar system.
% 
% The time evolution of protocluster cloud clumps provides a powerful constraint
% on models of both high-mass star and cluster formation.  In particular, the
% relative lifetimes of the starless and star-rich but dusty phases determines
% how effective feedback is at modifying gas properties and evacuating gas from a
% cluster \citep[e.g.,][]{Ginsburg2016b}.  Galactic plane surveys of dust emission
% have led to significant advances, constraining the timescale of the densest
% phases to be $<1$ Myr \citep{Svoboda2016a,Ginsburg2012a}, but these
% surveys are still affected by free-free contamination and therefore significant
% mass uncertainties.
% 
% Recent ground-based surveys have resulted in high-resolution ($\sim10\arcsec$)
% maps of large areas of the Galactic plane \citep[e.g.,][]{Lin2016a}.  While
% these surveys yield a much more detailed view of the morphological differences
% - and in some cases, evolutionary differences - between different
% protoclusters, they lack long-wavelength counterparts required to model the
% total dust mass and the free-free contamination.


%One of the earliest phases in a young massive star’s life cycle is to create an
%optically thick hypercompact HII (\hchii) region on small ($<500$ AU) scales.
%\MUSTANG is the best instrument available to survey for such objects because of
%its fast mapping speed and high sensitivity in a regime where dust and diffuse
%HII regions are both relatively faint, while \hchii emission is expected to be
%bright.  

We propose a survey of the 20 most massive proto-clusters in the northern
Galactic plane at 3 mm with \MUSTANG.  These dusty sources represent the
youngest observable stage in the formation of massive clusters and the next
generation of feedback engines in our Galaxy.  \MUSTANG observations will
determine the total dust mass in the proto-clusters.   These observations will
address how dust mass measurement is affected by free-free contamination of
thermal dust emission at shorter wavelengths.  With accurate dust measurements,
we can determine how far star formation has progressed in high-mass
protocluster-containing clouds, providing a measurement of the instantaneous
star formation efficiency and the massive cluster formation timescale.


\indent\underline{\textbf{\helv Dust as a mass estimator in high-mass protoclusters:}} 
%{\color{red}{\bf to be axed:} Some HII regions have shown evidence for
%excess microwave emission in the 15-30 GHz range, which has been attributed to
%electric dipole emission from spinning dust \citep{Watson2005a,Dickinson2007a}.
%There are claims that 50\% or more of the continuum emission in HII regions at
%wavelengths from 1 to 3 cm is due to spinning dust \citep{Todorovic2010a}.
%Since the microwave spinning dust signal is thought to originate from the same
%small-grain population responsible for the 24 \um (out of equilibrium,
%stochastically heated) thermal dust emission, the 90 GHz
%information will add a powerful constraint on the spectrum of spinning dust
%\citep{Draine1998a}.
%}
Accurate mass measurements are a critical first step toward understanding cloud
structure, mass budgets, and the star formation process.  In local clouds,
which lack massive stars and therefore free-free emission and can be observed
at high-spatial resolution from space, accurate mass measurements have been
enabled by Herschel.  For more massive clouds in the Galactic plane, free-free
contamination and spatial confusion significantly hamper mass measurements and
can easily lead to wrong conclusions about the density structure of clouds.
\MUSTANG observations can resolve these issues.

Star forming clouds concentrate their mass into progressively higher density
regions under the influence of gravity until stars are formed and their
feedback halts collapse.  The relative amount of mass at each column density,
often called the gas `PDF' (probability density function), has become a popular
mechanism for characterizing cloud evolutionary state.  This tool and others
\citep[e.g., column autocorrelation functions,][]{Lin2016a}  have been exploited
to measure the progression of star formation in clouds, but all are affected by
free-free contamination biasing the high-column measurements to regions that
are merely bright rather than massive.  Accurate measurements of these
region-characterizing statistics will, when combined with the high-resolution
maps illustrating the collapsing or chaotic morphology, definitively
characterize the evolutionary stage of these protoclusters.

%Spectral energy distribution (SED) fitting of Herschel dust maps have greatly
%improved measurements of dust masses.  Their main limitations are the short
%wavelength coverage (limited to $<500$ \um) and poor resolution at the longest
%wavelengths ($\sim40\arcsec$). The longest wavelength data are the most
%reliable for estimating the mass, since they are relatively unaffected by
%temperature uncertainties, but they become progressively more affected by
%contamination with free-free emission from ionized gas.

The cold dust observed from $\sim100$ \um to 3 mm is an accurate tracer of
the total mass of gas `clumps' that make up the densest part of molecular clouds.
The 3 mm data provide sensitivity to the coldest dust, being well within the
Rayleigh-Jeans limit for any temperatures so far observed.
%However, the accuracy of mass estimation is
%limited by uncertainties in the temperature and emissivity
%of the dust and, at long wavelengths, the amount of free-free “contamination”.
Long-wavelength data are necessary for estimating the dust emissivity power-law
index $\beta$, and the \MUSTANG 3mm data point is the longest available on
angular scales comparable to the (sub)millimeter instruments.
Assuming standard emissivity, a 0.5 mJy RMS is equivalent to 3.8\ee{21}
\persc, so our target $5\sigma$ sensitivity is $\sim1.9\ee{22}$ \persc, well
below the column density typically associated with high-mass star formation
\citep[e.g.,][]{Krumholz2008a}.
%(e.g., Krumholz \& McKee 2008).

To enable SED fitting, we will use SCUBA 450\um and SHARC/ARTEMIS 350\um data
that have recently become available at $\sim8-10\arcsec$ resolution toward many
high-mass star-forming regions \citep{Lin2016a}.  We are also obtaining 1100\um
data from LMT AzTEC  (PI: G{\'a}lvan-Madrid) at $\sim10$\arcsec resolution.
Together with the \MUSTANG data, we will obtain pixel-by-pixel SED fits and
column density measurements across the targeted reigons, \emph{including} a
free-free component that will be particularly well-constrained by the 3mm data
point.  These maps will be the highest-resolution column density maps so far
obtained in the Galactic plane, and they will be robust against temperature
variations and free-free contamination.

%Additionally, LMT 1100\um observations of many of these clusters has begun (PI:
%G{\'a}lvan-Madrid).  Including the 10\arcsec resolution data from the LMT, we will
%be able to create column density maps at this resolution that are robust to
%both temperature and free-free emission variations.



\underline{\textbf{\helv Measuring the population of young massive stars:}}
This pilot program will include the
first deep and broad search for massive protostars just now reaching the Zero
Age Main Sequence (ZAMS).  Deeply embedded, rapidly accreting young massive stars have
SED peaks in the millimeter regime and are faint at cm wavelengths because they
are compact and optically thick.  At shorter wavelengths, they are
indistinguishable from luminous dust clumps.  Individual rapidly accreting
stars in \hchii regions will be detectable across the galactic disk with
\MUSTANG over a full range of theoretical models.

Massive stars form in the middle of ultra-dense cores undergoing gravitational
collapse at a rate of order dM/dt $\sim 10^{-3}$ \msun \peryr such that a 100
\msun star takes about $10^5$ years to accrete its mass.  During their main
accretion phase, the photosphere will be bloated and the star will resemble a
red giant \citep{Yorke2002a,Hosokawa2009a}.    As the star contracts onto the
main sequence it starts to ionize its environment to create an \hchii region.
For a sufficiently dense accretion flow, or protostellar core, the Stromgren
radius of the \hchii regions will be bound by the gravity of the star, with a
radius $R_G \sim$50-100 AU. \citep{Keto2002a,Keto2003a,Keto2007a}.  These
gravitationally bound \hchii regions will be optically thick at centimeter
wavelengths therefore emit as blackbodies, with $S_\nu=21 \textrm{mJy} (d/5
\textrm{kpc})^2 (R/100 AU)^2(\nu/90 \textrm{GHz})^2$, which is only 0.06 mJy at
$\nu=5$ GHz, and therefore below the detection limit of existing surveys.
These \hchii regions will be bright at 90 GHz but faint at 5 GHz, which is a
direct means to distinguish them from older ultracompact (\uchii) regions.
Sources with a local peak at 3mm represent the youngest high-mass YSOs.

The clouds in our sample are the most massive in the Galaxy and the most likely
to be presently forming high-mass stars.  Our observations will be complete to
any high-mass young stars (HMYSOs) that have begun ionizing their surroundings,
even the very youngest (age $<1000$ years) that have just reach the ZAMS.  This
survey will therefore provide a complete sample of HMYSOs outside of cluster,
providing a unique sample for the study of the earliest phases of high-mass
star formation.



%Rotation breaks spherical symmetry, forcing the accretion to proceed via a much
%denser accretion disk with a photoionized surface. However, at $R < R_G$ the
%gravity of the star will confine the plasmas, creating a quasi-spherical \hchii
%region: even a massive star forming via accretion disk is expected, at its
%earliest stages, to be optically thick to free-free radiation.

%As long as this rapid accretion is ongoing, massive stars live in
%optically thick HII regions.
%\hchii regions will remain bound as long as
%accretion is ongoing. 
%The ram pressure of the inflow combined with the
%gravity of the star will prevent the thermal expansion of the region until
%winds, increasing ionizing luminosity of the young massive star, or other
%feedback terminates
%accretion by allowing the Str{\"o}mgren radius of the newly formed star to expand
%beyond $R_G$.  


% Assuming a massive cluster formation timescale of
% 0.3 Myr \citep[the upper limit given by studies of dusty proto-massive clusters by]
% []{Ginsburg2012a,Urquhart2013a,Urquhart2014a} and an expected 45 stars with $M>15$ \msun
% in a $10^4$ \msun cluster for a standard IMF, in a survey of 10 massive cluster forming regions, we
% would expect at least 1 star with `age' $<1000$ years.  If the progression from collapsing clump to ZAMS ionizing star is always linear, we would expect to see at least 1 source, but likely $\gtrsim10$, that is just beginning the process of igniting an HII region.  In reality, the process is unlikely to be so linear, since accretion rates vary \citep[e.g.][]{Galvan-Madrid2008a}, so we are instead likely to detect 10's of high-mass stars embedded within extremely dense environments, some of which will be very young and others continuing to accrete on the main sequence.  In either case, followup at high resolution will be critical to confirm the nature of these sources.  This survey will provide the best sample of extremely young high-mass YSOs for such followup.

%\hchii regions have been detected as unresolved sources with the VLA (e.g.
%Sewilo et al 2011), however most of these become optically thin in the 10-50
%GHz regime and are detected in the infrared, indicating that they are no longer
%bound, optically thin regions.  These thin \hchii regions represent the next
%stage in a young massive star’s life cycle and should be abundant.

\indent\underline{\textbf{\helv Synergy with other surveys \& telescopes}} 
The 3mm data break important degeneracies in the spectral energy distributions
(SEDs). There are currently no high-resolution ($<30\arcsec$), large-area
surveys between 5 GHz and 270 GHz.  At shorter and longer wavelengths our
survey will benefit from a wealth of currently available and ongoing surveys,
including: MAGPIS \citep{Helfand2006a}, CORNISH \citep{Hoare2012a}, ATLASGAL
\citep{Schuller2009a},  HIGAL \citep{Molinari2010a}, and the BGPS
\citep[][]{Aguirre2011a,Ginsburg2013a}.  More recently, there are incomplete
surveys that reach $\sim10\arcsec$ resolution at 350 \um with ARTEMIS and SHARC
\citep[e.g.][]{Lin2016a}, at 450 and 850 \um the JPS with SCUBA-2 on the JCMT,
and at 1100\um with AzTEC on the LMT. These surveys provide the data needed to
completely measure a dust + free-free SED.  The MUSTANG Galactic Plane Survey
(MGPS) will cover the largest area the fastest out of these new surveys
(excepting the JPS), and will therefore motivate the expansion and completion
of the short-wavelength surveys.

The VLA 5 GHz Galactic Plane Survey CORNISH, the most sensitive \uchii-region
survey to date, reached a typical RMS sensitivity of 0.4 mJy, yielding a
sensitivity limit of 3.9 mJy.  Any optically thick sources detected by CORNISH
would be at least 1300 mJy at 3.3 mm.  \MUSTANG’s $5-\sigma$ limit will be able
to detect optically thick sources down to the equivalent $S_{6cm} = 9$ $\mu$Jy,
comparable to the best a JVLA galactic plane survey could reasonably achieve
(though the JVLA would require $\sim20$ hours per square degree).


Since the turnover frequency is roughly proportional to electron density,
surveys in the radio continuum (all in the 1 to 5 GHz range) are strongly
biased against the densest and youngest objects.  These long-wavelength surveys
will be used to exclude older, extended \uchii regions from our set of
candidate \hchii's.  The short wavelength surveys constrain the dust masses and
temperatures and will be used to determine the turnover frequency for detected
\hchii regions.

Finally, \MUSTANG observations provide the only hope of acquiring zero-spacing
continuum information for ALMA 3mm observations.  Our observations will reach a
brightness temperature sensitivity of 1 mK over a square degree in only an hour
of observing with better resolution than the ALMA ACA, which would require
$\sim100$ hours to cover a square degree!  As part of this pilot, we will
demonstrate the combination of GBT continuum with a large (150 pointing,
6\arcmin$\times$6\arcmin) ALMA 3mm mosaic toward Sgr B2 (PI: Ginsburg).

\indent\underline{\textbf{\helv Target Selection:}} We have selected 11 target regions,
each one square degree since that will be the tile size for a future GPS, in the range $0\arcdeg < \ell < 50\arcdeg$
that encompass all of the 20 most massive proto-cluster “clumps” in the
northern Galactic plane \citep{Ginsburg2012a}.  Most of these are associated
with dozens of other nearby “clumps” of more moderate mass, and all are
associated with massive Giant Molecular Clouds (GMCs).  

The sample selected has already been observed at $\sim$30\arcsec resolution in
the sub-millimeter with Hi-Gal and the BGPS, at 20\arcsec with ATLASGAL, and at
10\arcsec with JCMT SCUBA-2 and CSO SHARC \citep{Lin2016a}.  
These regions are among the richest in the Galaxy and are therefore optimal for testing the performance of \MUSTANG in the Galactic plane and achieving our key science goals:
\vspace{-2mm}
\begin{enumerate}
    \item {\helv They are the most likely to have masses overestimated due to free-free
“contamination” in long-wavelength data (BGPS, ATLASGAL). 
3mm measurements will provide tight limits on the free-free contribution.
%, since even
%optically thick free-free emission has a shallower spectral index than dust,
%which will rise as $S_{dust}\propto\nu^{3.5}$ in the mm regime
}



    \item {\helv The host massive clouds are the most likely regions to contain the next
generation of massive star formation, and therefore the most likely to contain
young \hchii regions.  }

    \item {\helv \MUSTANG’s mass sensitivity will be comparable to other Galactic plane surveys assuming the
dust has an emissivity power-law of $\nu^{3.5}$, which is typical of Galactic plane
dust.  These data set will be used together to accurately measure the column density and, where possible, measure the dust emissivity power-law $\beta$.}
\end{enumerate}

\indent\underline{\textbf{\helv Sensitivity:}}  Our sensitivity goal is to
detect a 100 AU radius optically thick \hii region out to 15 kpc, so a
$5-\sigma$ detection limit of 2.5 mJy.  This sensitivity limit means our survey
will detect extremely young massive stars residing in bound \hii regions down
to $M \gtrsim 15$ \msun throughout our sample.  This sensitivity corresponds to
$N(\hh)=1.9\ee{22}$ \persc for $T=20$ K gas, comparable to that achieved by the
other Galactic plane surveys at this resolution.


%\MUSTANG’s 8\arcsec resolution is crucial to avoid confusion with dust emission.
%At the target RMS$=$0.5 mJy/beam sensitivity, \MUSTANG should detect any
%dust visible in previous surveys (BGPS, ATLASGAL).  At 8\arcsec resolution, \MUSTANG can detect individual
%proto-massive-star systems and distinguish them from their dusty or ionized surroundings.
%MUSTANG-2's projected mapping speed of $62 \mu$Jy/beam over $5^{\prime} \times 5^{\prime}$ 
%translates into 1.0 hour to cover one square degree to a sensitivity of 0.5 mJy/beam (assuming the fast-scanning mode is used, reducing noise by $35\%$). 
%\textbf{We aim to observe 11 square degrees with uniform sensitivity of 0.5 mJy/beam and therefore 
%request 31 hours of observation, including calibration time and overheads.}


% REFERENCES: Aguirre et al., ApJS 192, 1 (2011) * Draine & Lazarian ApJ 508, 157
% (1998) *  Dickinson et al. MNRAS 379, 297 (2007) * Ginsburg et al ApJL 758,L29
% (2012) * Green BASI 37, 45 (2009) * Helfand et al., AJ 131, 2525 (2006) *
% Hunter et al. ApJ 680, 1288 (2008) * Jackson et al., ApJ 163, 145 (2006) *
% Molinari et al. A&A 518, L100 (2010) * Murray & Rahman ApJ 709,424 (2010) *
% Sadler et al., MNRAS 385, 1656 (2008) * Schuller et al. A&A 504, 514 (2009) *
% Sewilo et al ApJS 194,44 (2011) * Shetty et al ApJ 696, 2234 (2009) * Shetty et
% al ApJ 696, 676 (2009) * Todorovic et al. MNRAS 406, 1629 (2010) * Watson et
% al. ApJ 624, L89 (2005) * Wood & Churchwell ApJ 340, 265(1989) * Hosokawa, T.,
% Omukai, K, 2009, ApJ, 691, 823-846 * Keto, E. 2007, ApJ, 666, 976-981 * Keto,
% E. 2003. ApJ, 599, 1196-1206 * Keto, E. 2002. ApJ, 580, 980-986 * Keto, E.
% 2002. ApJ, 568, 754-760  * Metzger, P. G, Henderson, A. P. 1967, ApJ, 147, 471
% * Yorke, H. W., Sonnhalter, C. 2002. Ap. J. 569:84662  * Urquhart et al 2014,
% MNRAS, 437, 1791-1807

\begin{figure}
\includegraphics[width=12cm]{coverage_gc.pdf}
\includegraphics[width=5.5cm]{MUSTANGwithSCUBA450contoursB.png}
\caption{(A) The target regions superposed on an emission map from the BGPS
\citep{Aguirre2011a,Ginsburg2013a}.  The red boxes represent the approximate
area to be covered by \MUSTANG.  The labels indicate source names \& distances
for the massive proto-clusters being targeted.  The observed 1-square-degree
regions include the entire parent molecular clouds of the massive clumps.  The
1-degree size scale is selected because \MUSTANG is very efficient at
covering maps of this size, and a Galactic plane survey can be carried out with
the same configuration.  The large maps are essential to include the complete
molecular clouds, since massive star formation is not isolated to the cluster
regions. \hspace{\textwidth}
(B) Grayscale image showing some of the first MUSTANG-1.5 data taken in the
Galactic plane toward the W51 star-forming region.  The image shows MUSTANG-1.5
data with RMS $\sim10$ mJy, much higher than our target sensitivity.  The
contours show SCUBA-2 450 \um emission, tracing the warm dust.  In this case,
the 3mm emission is dominated by free-free emission on the largest scales, but
the compact sources e2 and IRS2 both contain high-mass protostars, as has
been demonstrated with VLA and ALMA observations \citep{Ginsburg2016b}.
}
\label{fig:figure}
\end{figure}

\clearpage

NOTE: THIS DOCUMENT IS INTENDED TO BE CUT HERE.  Only the first 4 pages are to be submitted.


% \Figure{coverage_gc.pdf}
% {The target regions superposed on an emission map from the BGPS
% \citep{Aguirre2011a,Ginsburg2013a}.  The red boxes represent the approximate
% area to be covered by \MUSTANG.  The labels indicate source names \& distances
% for the massive proto-clusters being targeted.  The observed 1-square-degree
% regions include the entire parent molecular clouds of the massive clumps.  The
% 1-degree size scale is selected because \MUSTANG is very efficient at
% covering maps of this size, and a Galactic plane survey can be carried out with
% the same configuration.  The large maps are essential to include the complete
% molecular clouds, since massive star formation is not isolated to the cluster
% regions.}
% {fig:coverage}{0.3}{0}
% 
% 
% 
% \Figure{rKuVLA_gMUSTANG_bSCUBA450.png}
% {RGB image showing some of the first MUSTANG-1.5 data taken in the Galactic plane toward the W51 star-forming region.  Red is JVLA Ku-band, green is MUSTANG-1.5 (with RMS $\sim10$ mJy, much higher than our target), and blue is SCUBA-2 450um.  In this bright extended HII region, the 3mm emission is dominated by free-free emission, }
% {fig:mustangRGB}{0.25}{0}


%\textbf{Final note:} This is a resubmission of the A-ranked proposal
%GBT/14A-329 (linear rank score 0.82), which was not observed because of instrument delays, and
%GBT/15A-172.  We have addressed the concerns expressed in the latter review.
%Our sources are only observable for a portion of this proposal cycle
%(August-October), so completion may require either receiving
%an A-ranking or a resubmission next cycle.

\clearpage
{\color{red} Bibliographies take space and leaving them out doesn't seem to cause complaints...}
%\begin{multicols}{3}
%\begin{spacing}{0.5}
\footnotesize\raggedright
\noindent 
%\vskip-15pt
\bibliography{bibdesk}
\normalsize
%\vskip-15pt
%\end{spacing}
%\end{multicols}
%\bibliography{bibdesk}


ABSTRACT:
We propose a pilot program for a Galactic Plane survey with MUSTANG.  We will
survey the most massive proto-clusters in the Galaxy, aiming to accurately
determine dust-based column densities at 8 arcsecond resolution while searching
for extremely dense hypercompact HII (HCHII) regions that represent a very
early stage of high-mass star formation.  We will measure the full SEDs of the
proto-clusters by combining MUSTANG-2 with SCUBA, SHARC, AzTEC, and SPIRE data in
order to accurately measure their total dust mass and ionizing flux
density. These data will place strong constraints on the dust emissivity index,
allowing for the best possible measurements of dust and gas mass. 
They will provide the highest-resolution column density maps achievable that
are robust against temperature variation and free-free contamination.
We will obtain a unique sample of very young high-mass protostars.  Finally,
we will demonstrate the combination of GBT MUSTANG data with ALMA continuum
data, filling in its missing short information.

\end{document}
